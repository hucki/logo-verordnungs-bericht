\documentclass[12pt,a4paper]{article}
\usepackage{graphicx}
\usepackage{hyperref}
\usepackage{pifont}
\hypersetup{pdfauthor=Stefan Huckschlag, pdftitle=Verordnungs-Bericht, pdfsubject={Dokumentenvorlage gemäß Anhang A zu Anlage 1 zum Vertrag nach § 125 Absatz 1 SGB V: Verordnungs-Bericht, Bericht gemäß § 13 Absatz 2 lit. d, § 16 Absatz 7 HeilM-RL / § 11 Absatz 2 lit. c, § 15 Absatz 5 HeilM-RL ZÄ}, pdfkeywords={reich werden, Geld drucken}}
\usepackage[absolute]{textpos}
\thispagestyle{empty}
\setlength{\TPHorizModule}{1mm}
\setlength{\TPVertModule}{\TPHorizModule}
\setlength{\parindent}{0pt}
\title{Verordnungs-Bericht}
\def\cb#1{\CheckBox[%
width=2.2mm,%
height=2.2mm,%
checkboxsymbol=5,%
bordercolor=0 0 0,%
charsize=2.1mm,%
name=ch#1]{\rule{0mm}{0mm}}}
\begin{document}
  \begin{textblock}{1000}(0,70)
    % \includegraphics[width=\paperwidth,trim=left bottom right top, clip]{20201214_Anlage_1_Anhang_A_Ergebnis.pdf}
    \includegraphics[width=\paperwidth, trim=0 60mm 0 30mm, clip]{20201214_Anlage_1_Anhang_A_Ergebnis.pdf}
  \end{textblock}
  % \begin{Form}[action=mailto:info@example.com,encoding=html,method=post]
  \begin{Form}
% Personalien der oder des Versicherten
    \begin{textblock}{1000}(26.4,92)
      \TextField[%
      width=75mm,%
      height=30mm,%
      multiline=true,%
      align=0,%
      bordercolor=1 1 1,%
      borderwidth=0mm,%
      charsize=0pt, %
      name=txPerson]{\rule{0mm}{0mm}}
    \end{textblock}
% Datum / Gruppe Diagnose
    \begin{textblock}{1000}(141,85)
      \TextField[%
      width=40mm,%
      height=5mm,%
      multiline=false,%
      align=1,%
      bordercolor=1 1 1,%
      borderwidth=0mm,%
      charsize=0pt,%
      name=txVerordnungsdatum]{\rule{0mm}{0mm}}
    \end{textblock}
    \begin{textblock}{1000}(141,95.6)
      \TextField[%
      width=40mm,%
      height=5mm,%
      multiline=false,%
      align=1,%
      bordercolor=1 1 1,%
      borderwidth=0mm,%
      charsize=0pt,%
      name=txDiagnosegruppe]{\rule{0mm}{0mm}}
    \end{textblock}
    \begin{textblock}{1000}(106,113)
      \TextField[%
      width=75mm,%
      height=9mm,%
      multiline=true,%
      align=1,%
      bordercolor=1 1 1,%
      borderwidth=0mm,%
      charsize=0pt,%
      name=txTherapeutischeDiagnose]{\rule{0mm}{0mm}}
    \end{textblock}
% Empfehlungen der Therapeutin oder des Therapeuten
  % linke Spalte
    \begin{textblock}{1000}(26.4,134.4)
      \cb{Fortfuehrung}
    \end{textblock}
    \begin{textblock}{1000}(26.4,140.4)
      \cb{Therapiepause}
    \end{textblock}
    \begin{textblock}{1000}(26.4,146.4)
      \cb{Beendigung}
    \end{textblock}
    \begin{textblock}{1000}(26.4,152.4)
      \cb{Wiedervorstellung}
    \end{textblock}
    \begin{textblock}{1000}(66.4,149.9)
      \TextField[%
      width=20mm,%
      height=4mm,%
      multiline=false,%
      align=1,%
      bordercolor=1 1 1,%
      borderwidth=0mm,%
      charsize=0pt,%
      name=txWiedervorstellungInWochen]{\rule{0mm}{0mm}}
    \end{textblock}
    \begin{textblock}{1000}(26.4,158.4)
      \cb{andereTherapie}
    \end{textblock}
    \begin{textblock}{1000}(61.4,155.9)
      \TextField[%
      width=37mm,%
      height=4mm,%
      multiline=false,%
      align=0,%
      bordercolor=1 1 1,%
      borderwidth=0mm,%
      charsize=0pt,%
      name=txWelcheAndereTherapie]{\rule{0mm}{0mm}}
    \end{textblock}
  % rechte Spalte
  \begin{textblock}{1000}(106.2,134.4)
    \cb{Einzeltherapie}
  \end{textblock}
  \begin{textblock}{1000}(157,131.9)
    \TextField[%
    width=24mm,%
    height=4mm,%
    multiline=false,%
    align=1,%
    bordercolor=1 1 1,%
    borderwidth=0mm,%
    charsize=0pt,%
    name=txEinzeltherapieMinuten]{\rule{0mm}{0mm}}
  \end{textblock}
  \begin{textblock}{1000}(106.2,140.4)
    \cb{Gruppentherapie}
  \end{textblock}
  \begin{textblock}{1000}(157,137.9)
    \TextField[%
    width=24mm,%
    height=4mm,%
    multiline=false,%
    align=1,%
    bordercolor=1 1 1,%
    borderwidth=0mm,%
    charsize=0pt,%
    name=txGruppentherapieMinuten]{\rule{0mm}{0mm}}
  \end{textblock}
  \begin{textblock}{1000}(106.24,146.4)
    \cb{Doppelbehandlung}
  \end{textblock}
  \begin{textblock}{1000}(106.2,152.4)
    \cb{Frequenz}
  \end{textblock}
  \begin{textblock}{1000}(157,149.9)
    \TextField[%
    width=24mm,%
    height=4mm,%
    multiline=false,%
    align=1,%
    bordercolor=1 1 1,%
    borderwidth=0mm,%
    charsize=0pt,%
    name=txAnzahlProWoche]{\rule{0mm}{0mm}}
  \end{textblock}
  \begin{textblock}{1000}(106.2,158.4)
    \cb{Hausbesuch}
  \end{textblock}
% Begründung
  \begin{textblock}{1000}(26.2,171.4)
    \TextField[%
    width=75mm,%
    height=50mm,%
    multiline=true,%
    align=0,%
    bordercolor=1 1 1,%
    borderwidth=0mm,%
    charsize=0pt, %
    name=txBegruendung]{\rule{0mm}{0mm}}
  \end{textblock}
  \begin{textblock}{1000}(106.2,170)
    \TextField[%
    width=40mm,%
    height=5mm,%
    multiline=false,%
    align=1,%
    bordercolor=1 1 1,%
    borderwidth=0mm,%
    charsize=0pt,%
    name=txDatumDesBerichts]{\rule{0mm}{0mm}}
  \end{textblock}
    \begin{textblock}{1000}(26.4,230.4)
      \small{Bericht gemäß § 13 Absatz 2 lit. d, § 16 Absatz 7 HeilM-RL /\\
             § 11 Absatz 2 lit. c, § 15 Absatz 5 HeilM-RL ZÄ}
      % \Acrobatmenu{Print}{Print}
      % \Submit[export=html]{Submit}
      % \Reset{Clear}
    \end{textblock}
  \end{Form}
\end{document}